\documentclass[journal,12pt,twocolumn]{IEEEtran}

\usepackage{enumitem}
\usepackage{amsmath}
\usepackage{amssymb}
\usepackage{gensymb}
\usepackage{graphicx}
\usepackage{txfonts}         
\usepackage{listings}
\usepackage{lstautogobble}
\usepackage{mathtools}
\usepackage{bm}
\usepackage{hyperref}
\usepackage{polynom}
\usepackage{capt-of}
\newcommand{\solution}{\noindent \textbf{Solution: }}
\providecommand{\pr}[1]{\ensuremath{\Pr\left(#1\right)}}
\providecommand{\brak}[1]{\ensuremath{\left(#1\right)}}
\providecommand{\cbrak}[1]{\ensuremath{\left\{#1\right\}}}
\providecommand{\sbrak}[1]{\ensuremath{\left[#1\right]}}
\providecommand{\mean}[1]{E\left[ #1 \right]}
\providecommand{\var}[1]{\mathrm{Var}\left[ #1 \right]}
\providecommand{\der}[1]{\mathrm{d} #1}
\providecommand{\gauss}[2]{\mathcal{N}\ensuremath{\left(#1,#2\right)}}
\providecommand{\mbf}{\mathbf}
\providecommand{\abs}[1]{\left\vert#1\right\vert}
\providecommand{\norm}[1]{\left\lVert#1\right\rVert}
\providecommand{\z}[1]{{\mathcal{Z}}\{#1\}}
\providecommand{\ztrans}{\overset{\mathcal{Z}}{ \rightleftharpoons}}

\providecommand{\parder}[2]{\frac{\partial}{\partial #2} \brak{#1}}

\let\StandardTheFigure\thefigure
\let\vec\mathbf

\numberwithin{equation}{section}
\renewcommand{\thefigure}{\theenumi}
\renewcommand\thesection{\arabic{section}}

\newcommand{\myvec}[1]{\ensuremath{\begin{pmatrix}#1\end{pmatrix}}}
\newcommand{\mydet}[1]{\ensuremath{\begin{vmatrix}#1\end{vmatrix}}}
\newcommand{\define}{\stackrel{\triangle}{=}}

\DeclareMathOperator*{\argmin}{arg\,min}
\DeclareMathOperator*{\argmax}{arg\,max}


\lstset {
	frame=single, 
	breaklines=true,
	columns=fullflexible,
	autogobble=true
}             
   


\begin{document}
                             
\title{ Digital Signal Processing \\ \Large EE3900: Linear Systems and Signal Processing \\ \large Indian Institute of Technology Hyderabad \\ \vspace*{12pt} \textbf{Assignment-1}}
\author{Lokesh Badisa \\ \normalsize AI21BTECH11005 \\ \vspace*{20pt} \normalsize 10 Oct 2022}   
 \maketitle 
 \tableofcontents
 \begin{abstract}
 This document contains solution to Assignment-1 [ Question 3.1(f) from Discrete-Time Signal Processing by Alan V. Oppenheim and Ronald W. Schafer]
 \end{abstract}
\section{JEE 2019}
Let 
\begin{align}
	a_n &= \frac{\alpha^{n}-\beta^{n}}{\alpha - \beta}, \quad n \ge 1
	\\
	b_n &= a_{n-1} + a_{n+1}, \quad n \ge 2, \quad b_1 =1
	\label{eq:10-orig-diff}
\end{align}
Verify the following using a python code.
\begin{enumerate}[label=\thesection.\arabic*
,ref=\thesection.\theenumi]
\item 
\begin{align}
	\sum_{k=1}^{n}a_k = a_{n+2}-1, \quad n \ge 1
\end{align}
 \item 
\begin{align}
	\sum_{k=1}^{\infty}\frac{a_k}{10^k} =\frac{10}{89}
\end{align}
 \item 
\begin{align}
	b_n =\alpha^n + \beta^n, \quad n \ge 1
\end{align}
 \item 
\begin{align}
	\sum_{k=1}^{\infty}\frac{b_k}{10^k} =\frac{8}{89}
\end{align}

\end{enumerate}
\section{Pingala Series}
\begin{enumerate}[label=\thesection.\arabic*,ref=\thesection.\theenumi]
\item The {\em one sided} $Z$-transform of $x(n)$ is defined as 
%\cite{proakis_dsp}
\begin{align}
	X^{+}(z) = \sum_{n = 0}^{\infty}x(n)z^{-n}, \quad z \in \mathbb{C}
\label{eq:one-Z}
\end{align}
	\item The {\em Pingala} series is generated using the difference equation 
\begin{align}
	x(n+2) = x\brak{n+1} + x\brak{n},  \quad x(0) = x(1) = 1, n \ge 0
	\label{eq:10-pingala}
\end{align}
Generate a stem plot for $x(n)$.
\item 		Find $X^{+}(z)$.
	\item Find $x(n)$.

	\item Sketch 
\begin{align}
	y(n)	 = x\brak{n-1} + x\brak{n+1},  \quad n \ge 0
	\label{eq:10-orig-diff-rev}
\end{align}

	\item Find $Y^{+}(z)$. 
\item Find $y(n)$.
\end{enumerate}
\section{Power of the Z transform}
\begin{enumerate}[label=\thesection.\arabic*,ref=\thesection.\theenumi]
\item Show that 
\begin{align}
	\sum_{k=1}^{n}a_k = 
	\sum_{k=0}^{n-1}x(n) = x(n)*u(n-1)
\end{align}
\item Show that 
\begin{align}
a_{n+2}-1, \quad n \ge 1
\end{align}
can be expressed as 
\begin{align}
	\sbrak{x\brak{n+1}-1}u\brak{n}
\end{align}
 \item Show that 
\begin{align}
	\sum_{k=1}^{\infty}\frac{a_k}{10^k}= 
	\frac{1}{10}\sum_{k=0}^{\infty}\frac{x\brak{k}}{10^k} =\frac{1}{10}X^{+}\brak{{10}}
\end{align}
 \item Show that 
\begin{align}
	\alpha^n + \beta^n, \quad n \ge 1
\end{align}
can be expressed as 
\begin{align}
	w(n) =\brak{\alpha^{n+1} + \beta^{n+1}}u(n)
\end{align}
		and find $W(z)$.

 \item Show that 
\begin{align}
	\sum_{k=1}^{\infty}\frac{b_k}{10^k} =
	\frac{1}{10}\sum_{k=0}^{\infty}\frac{y\brak{k}}{10^k} =\frac{1}{10}Y^{+}\brak{{10}}
\end{align}
\item Solve the JEE 2019 problem.

\end{enumerate}


\end{document}
