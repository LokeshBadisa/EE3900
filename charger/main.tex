\documentclass[journal,12pt,twocolumn]{IEEEtran}

\usepackage{enumitem}
\usepackage{amsmath}
\usepackage{amssymb}
\usepackage{gensymb}
\usepackage{graphicx}
\usepackage{txfonts}         
\usepackage{listings}
\usepackage{lstautogobble}
\usepackage{mathtools}
\usepackage{bm}
\usepackage{hyperref}
\usepackage{polynom}
\usepackage{capt-of}
\newcommand{\solution}{\noindent \textbf{Solution: }}
\providecommand{\pr}[1]{\ensuremath{\Pr\left(#1\right)}}
\providecommand{\brak}[1]{\ensuremath{\left(#1\right)}}
\providecommand{\cbrak}[1]{\ensuremath{\left\{#1\right\}}}
\providecommand{\sbrak}[1]{\ensuremath{\left[#1\right]}}
\providecommand{\mean}[1]{E\left[ #1 \right]}
\providecommand{\var}[1]{\mathrm{Var}\left[ #1 \right]}
\providecommand{\der}[1]{\mathrm{d} #1}
\providecommand{\gauss}[2]{\mathcal{N}\ensuremath{\left(#1,#2\right)}}
\providecommand{\mbf}{\mathbf}
\providecommand{\abs}[1]{\left\vert#1\right\vert}
\providecommand{\norm}[1]{\left\lVert#1\right\rVert}
\providecommand{\z}[1]{{\mathcal{Z}}\{#1\}}
\providecommand{\ztrans}{\overset{\mathcal{Z}}{ \rightleftharpoons}}
\providecommand{\system}[1]{\overset{\mathcal{#1}}{ \longleftrightarrow}}
\providecommand{\laplaceinv}[1]{{\mathcal{L}^{-1}\ensuremath{\left[#1\right]}}}
\providecommand{\parder}[2]{\frac{\partial}{\partial #2} \brak{#1}}

\let\StandardTheFigure\thefigure
\let\vec\mathbf

\numberwithin{equation}{section}
\renewcommand{\thefigure}{\theenumi}
\renewcommand\thesection{\arabic{section}}

\newcommand{\myvec}[1]{\ensuremath{\begin{pmatrix}#1\end{pmatrix}}}
\newcommand{\mydet}[1]{\ensuremath{\begin{vmatrix}#1\end{vmatrix}}}
\newcommand{\define}{\stackrel{\triangle}{=}}

\DeclareMathOperator*{\argmin}{arg\,min}
\DeclareMathOperator*{\argmax}{arg\,max}


\lstset {
	frame=single, 
	breaklines=true,
	columns=fullflexible,
	autogobble=true
}             
   


\begin{document}
                             
\title{ Digital Signal Processing \\ \Large EE3900: Linear Systems and Signal Processing \\ \large Indian Institute of Technology Hyderabad \\ \vspace*{12pt} \textbf{Fourier Series}}
\author{Lokesh Badisa\\ \normalsize AI21BTECH11001 \\ \vspace*{20pt} \normalsize 12 Oct 2022  }
 \maketitle 
 \tableofcontents
 \begin{abstract}
    This manual provides a simple introduction to Fourier Series
    \end{abstract}
    \section{Periodic Function}
    Let 
    \begin{align}
        x(t) &= A_0\abs{\sin\brak{2\pi f_0 t}}
        \label{eq:10-orig-diff-def}
    \end{align}
    \begin{enumerate}[label=\thesection.\arabic*
    ,ref=\thesection.\theenumi]
    \item Plot $x(t)$.\\
    \solution 
    \begin{lstlisting}
        wget https://raw.githubusercontent.com/LokeshBadisa/EE3900-Linear-Systems-and-Signal-Processing/main/charger/codes/1.1.py
        python3 1.1.py
    \end{lstlisting}
    \begin{figure}[!ht]
			\centering
			\includegraphics[width=\columnwidth]{./figs/1.1}
			\caption{}
			%\label{fig:ckt}
\end{figure}
    \item Show that $x(t)$ is periodic and find its period.
    \solution 
    If a signal $x(t)$ is periodic then
    \begin{align}
    x(t+T)=x(t)
    \end{align}
    where $T$ is known as fundamental period.
    \indent Since $|sin\theta|$ function is periodic, x(t) is also periodic.
    \begin{align}
\text{Fundamental Period}=T=\frac{1}{2}\brak{\frac{2 \pi}{2 \pi f_0}}\\
\label{eq:ftp}
=\frac{1}{2f_0}
    \end{align}
    \end{enumerate}
    \section{Fourier Series}
    Consider $A_0 =12$ and $f_0 = 50$ for all numerical calculations.
    \begin{enumerate}[label=\thesection.\arabic*,ref=\thesection.\theenumi]
    \item If
    %\cite{proakis_dsp}
    \begin{align}
        x(t) = \sum_{k = -\infty}^{\infty}c_ke^{\j2\pi kf_0 t}
    \label{eq:one-Z-complex}
    \end{align}
    show that 
    \begin{align}
        c_k = f_0\int_{-\frac{1}{2f_0}}^{\frac{1}{2f_0}}x(t)e^{-\j2\pi kf_0 t}\, dt
    \label{eq:one-Z}
    \end{align}
    \solution 
    From \eqref{eq:one-Z-complex},
    \begin{align}
            x(t) = \sum_{k = -\infty}^{\infty}c_ke^{\j2\pi kf_0 t}
    \end{align}
    Mulitply $e^{-\j2\pi lf_0 t}$ on both sides
    \begin{align}
    x(t)e^{-\j2\pi lf_0 t}=\sum_{k = -\infty}^{\infty}c_ke^{\j2\pi kf_0 t}e^{-\j2\pi lf_0 t}
    \end{align}
    Integrate on both sides with respect to 't' between $-T$ to $T$ where T is fundamental time period of x(t).\\
    Using \eqref{eq:ftp},
    \begin{align}
    T=\frac{1}{2f_0}
    \end{align}
    \begin{align}
    \int_{-\frac{1}{2f_0}}^{\frac{1}{2f_0}}x(t)e^{-\j2\pi kf_0 t}\, dt=\int_{-\frac{1}{2f_0}}^{\frac{1}{2f_0}}\sum_{k = -\infty}^{\infty}c_ke^{\j2\pi \brak{k-l}f_0 t}\,dt\\
    =\sum_{k = -\infty}^{\infty}c_k\int_{-\frac{1}{2f_0}}^{\frac{1}{2f_0}}e^{\j2\pi \brak{k-l}f_0 t}\,dt
    \end{align}
    The above integral:
    \begin{align}
    \int_{-\frac{1}{2f_0}}^{\frac{1}{2f_0}}e^{\j2\pi \brak{k-l}f_0 t}\,dt=
          \begin{cases}
0 & k\neq l
\\
 \int_{-\frac{1}{2f_0}}^{\frac{1}{2f_0}}1\,dt& k=l
\end{cases}
    \end{align}
    \begin{align}
   \therefore \int_{-\frac{1}{2f_0}}^{\frac{1}{2f_0}}x(t)e^{-\j2\pi kf_0 t}\, dt &=\brak{\frac{1}{f_0}}c_k\\
  \therefore c_k = f_0\int_{-\frac{1}{2f_0}}^{\frac{1}{2f_0}}x(t)e^{-\j2\pi kf_0 t}\, dt
    \end{align}
        \item Find $c_k$ for 
        \eqref{eq:10-orig-diff-def}\\
        \solution
        $c_k$ can be calculated even simpler by using
        \begin{align}
        c_k = 2f_0\int_{0}^{\frac{1}{2f_0}}x(t)e^{- \j2 \pi kf_0 t}\, dt
        \end{align}
        $x(t)=A_0 \sin\brak{2 \pi f_0t}$ in 0 to $\frac{1}{2f_0}$ region.\\
Also,
        \begin{align}
        \label{eq:sin}
     \sin \theta  =\frac{e^{\j\theta}-e^{-\j \theta}}{2\j} 
      \end{align}
      Using \eqref{eq:sin},
      \begin{align}
      c_k &= 2f_0\int_{0}^{\frac{1}{2f_0}}A_0 \brak{\frac{e^{\j2\pi f_0t}-e^{-\j 2\pi f_0t}}{2\j}} e^{- \j2 \pi kf_0 t}\,dt\\
      &=A_0f_0\int_{0}^{\frac{1}{2f_0}} \brak{\frac{e^{\j2\pi\brak{1-k} f_0t}-e^{\j 2\pi\brak{-1-k} f_0t}}{\j}}\,dt\\
&=A_0f_0\sbrak{\frac{e^{\j2\pi\brak{1-k} f_0t}}{-2\pi \brak{1-k}f_0}\Big|_0^{\frac{1}{2f_0}} - \frac{e^{\j2\pi\brak{-1-k} f_0t}}{-2\pi \brak{-1-k}f_0}\Big|_0^{\frac{1}{2f_0}}}\\
&=A_0\sbrak{\frac{e^{j\pi\brak{1-k}}-1}{2\pi\brak{k-1}}-\frac{e^{-j\pi\brak{1+k}}-1}{2\pi\brak{k+1}}}
      \end{align}
      \begin{equation}
      \label{eq:ck}
     = \begin{cases}
\frac{2A_0}{\pi\brak{1-k^2}}&k=even
\\
0&k=odd
\end{cases}
      \end{equation}
    \item Verify 
        \eqref{eq:10-orig-diff-def}
        using python.\\
        \solution 
        \begin{lstlisting}
        wget https://raw.githubusercontent.com/LokeshBadisa/EE3900-Linear-Systems-and-Signal-Processing/main/charger/codes/2.3.py
        python3 2.3.py
        \end{lstlisting}
          \begin{figure}[!ht]
			\centering
			\includegraphics[width=\columnwidth]{./figs/2.3}
			\caption{}
			%\label{fig:ckt}
\end{figure}
         \item Show that 
    \begin{align}
        x(t) = \sum_{k = 0}^{\infty}\brak{a_k\cos{2\pi kf_0 t}+b_k\sin{2\pi kf_0 t}}
    \label{eq:one-Z-real}
    \end{align}
    and obtain the formulae for $a_k$ and $b_k$.\\
    \solution
    Using \eqref{eq:one-Z-complex},
    \begin{align}
    x(t) = \sum_{k = -\infty}^{\infty}c_ke^{\j2\pi kf_0 t}
    \end{align}
    As,
    \begin{align}
    e^{\j2\pi kf_0 t}=\cos\brak{2\pi kf_0 t}+j\sin\brak{2\pi kf_0 t}
    \end{align}
    Substituting leads to
    \begin{align}
    x(t) &= \sum_{k = -\infty}^{\infty}c_k\sbrak{\cos\brak{2\pi kf_0 t}+j\sin\brak{2\pi kf_0 t}}\\
    \label{eq:2.4}
    &=\sum_{k = -\infty}^{\infty}c_k\cos\brak{2\pi kf_0 t}+jc_k\sin\brak{2\pi kf_0 t}\\
    &=\sum_{k = -\infty}^{-1}\sbrak{c_k\cos\brak{2\pi kf_0 t}+jc_k\sin\brak{2\pi kf_0 t}}+c_0+\sum_{k = 1}^{\infty}\sbrak{c_k\cos\brak{2\pi kf_0 t}+jc_k\sin\brak{2\pi kf_0 t}}\\
    &=\sum_{k = 1}^{\infty}\sbrak{c_{-k}\cos\brak{2\pi kf_0 t}-jc_{-k}\sin\brak{2\pi kf_0 t}}+c_0+\sum_{k = 1}^{\infty}\sbrak{c_k\cos\brak{2\pi kf_0 t}+jc_k\sin\brak{2\pi kf_0 t}}\\
    &=c_0+\sum_{k = 1}^{\infty}\sbrak{\brak{c_k+c_{-k}}\cos\brak{2\pi kf_0 t}+j\brak{c_k-c_{-k}}\sin\brak{2\pi kf_0 t}}
    \end{align}
    Replacing $\brak{c_k+c_{-k}}\to a_k$ and $j\brak{c_k-c_{-k}} \to b_k$,
    \begin{align}
&=c_0+    \sum_{k = 1}^{\infty}\brak{a_k\cos{2\pi kf_0 t}+b_k\sin{2\pi kf_0 t}}\\
&=\sum_{k = 0}^{\infty}\brak{a_k\cos{2\pi kf_0 t}+b_k\sin{2\pi kf_0 t}}
    \end{align}
    \begin{align}
    \label{eq:u1}
    \therefore a_k=
    \begin{cases}
    c_k+c_{-k}&k\neq0
    \\
    c_0&k=0
    \end{cases}\\
    \label{eq:u2}
    b_k=j\brak{c_k-c_{-k}}
    \end{align}
    Using \eqref{eq:one-Z},
    \begin{align}
    c_k = f_0\int_{-\frac{1}{2f_0}}^{\frac{1}{2f_0}}x(t)e^{-\j2\pi kf_0 t}\, dt\\
    c_{-k} = f_0\int_{-\frac{1}{2f_0}}^{\frac{1}{2f_0}}x(t)e^{\j2\pi kf_0 t}\, dt\end{align}
    \begin{align}
    a_k=c_k+c_{-k}&= f_0\int_{-\frac{1}{2f_0}}^{\frac{1}{2f_0}}x(t)\sbrak{e^{-\j2\pi kf_0 t}+e^{\j2\pi kf_0 t}}\, dt\\
    &=2f_0\int_{-\frac{1}{2f_0}}^{\frac{1}{2f_0}}x(t)\cos\brak{2\pi kf_0t}\, dt
    \end{align}
    Parallely,
    \begin{align}
    b_k&=-2jf_0\int_{-\frac{1}{2f_0}}^{\frac{1}{2f_0}}x(t)\sin\brak{2\pi kf_0t}\, dt
    \end{align}
    \item Find $a_k$ and $b_k$ for 
        \eqref{eq:10-orig-diff-def}\\
        \solution
        Using \eqref{eq:u1} and \eqref{eq:u2} with \eqref{eq:ck},
    \begin{align}
    a_k=c_k+c_{-k}=\begin{cases}
\frac{4A_0}{\pi\brak{1-k^2}}&k=even
\\
\frac{2A_0}{\pi}&k=0
\\
0&k=odd
\end{cases}\\
b_k=j\brak{c_k-c_{-k}}=0
    \end{align}
    \item Verify 
    \eqref{eq:one-Z-real}
    using python.\\
     \solution 
        \begin{lstlisting}
        wget https://raw.githubusercontent.com/LokeshBadisa/EE3900-Linear-Systems-and-Signal-Processing/main/charger/codes/2.6.py
        python3 2.3.py
        \end{lstlisting}
          \begin{figure}[!ht]
			\centering
			\includegraphics[width=\columnwidth]{./figs/2.6}
			\caption{}
			%\label{fig:ckt}
\end{figure}
    \end{enumerate}
    \end{document}